% !TeX spellcheck = en_US
\chapter{Technologies}

\fxnote{All the ref's are missing}

Each technology used have been listed below, each was chosen based on the requirements, price and usability.

One of the main non-functional requirements was the use of Window 8 \& 10 platforms, therefor an obvious choice was to use C\# in corporation with WPF since it allows fast development.

\section{.Net}

.Net Version 4.6.1 was used since it was the newest available at the time of development, and offers increased performance in WPF-applications. 

\section{Windows Presentation Foundation}

WPF (Windows Presentation Foundation) Was used to develop the front-end, it's comes with .Net and was therefor an obvious choice.

\section{Dynamic Data Display}

D3 (Dynamic Data Display) is a WPF-Compatible plotter, used for plotting the incoming data. 

\section{Prism}

Prism is a collection of pattens often used in WPF applications.

\section{Entity framework}

\section{NUnit}

NUnit 2.* was used as the unit-test framework.

\section{NSubstitute}

NSubstitute was used as a mock framework when unit-testing.