% !TeX spellcheck = en_US
\chapter{Technologies}

Each technology used to develop the Rolling Road GUI have been listed below, each was chosen based on the requirements, price and usability.

One of the main non-functional requirements was the use of Window 8 \& 10 platforms, therefor an obvious choice was to use C\# in corporation with WPF since it allows fast development.

\section{.Net}

.Net Version 4.6.1 was used since it was the newest available at the time of development and offers increased performance in WPF-applications. 

\section{Windows Presentation Foundation}

WPF (Windows Presentation Foundation) Was used to develop the front-end, it's packaged with Visual Studio, making it easy to setup and developing in.

\section{Dynamic Data Display}

D3 (Dynamic Data Display) is a WPF-Compatible plotter/graph library, used for plotting the incoming data. It's a bit outdated, but it has features such as zooming and screenshots. Making it one of the better free plotting tools for WPF.

\section{Prism}

Prism is a collection of pattens such as the DelegateCommand often used in WPF applications using MVVM.

\section{Entity framework}

EF6 (Entity Framework 6) was to used to abstract away from the data source, enabling the use of a database later in the process if a change in requirements occur.

\section{NUnit}

NUnit 2.* was used as the unit-test framework.

\section{NSubstitute}

NSubstitute was used as a mock framework when unit-testing.